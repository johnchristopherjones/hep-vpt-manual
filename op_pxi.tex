\chapter{PXI Crate}
\label{sec:op_pxi}

\section{Logging into the PXI Crate (RDP)}
\label{sec:op_pxi:rdp}

\subsection{Mac}
\label{sec:op_pxi:rdp:mac}

You'll need to download and install Microsoft's \href{http://www.microsoft.com/mac/products/remote-desktop/default.mspx}{Remote Desktop Connection Client for Mac}.

\begin{enumerate}
\item Launch \menu{Remote Desktop Connection for Mac}.
\item In the ``Computer:'' field, enter the IP address \texttt{128.143.196.230}.  Press \menu{Connect}.
\item When prompted, use the username \texttt{administrator} and password \texttt{UVAVPT}
\end{enumerate}

If desired, you can make local (Mac) hard drives and printers available to the PXI Crate while you're logged in by editing the connection. (\menu{File \menusep Edit a Connection})

\section{Launching LabVIEW}
\label{sec:op_pxi:labview_launch}

\section{Opening Project \textit{VPT Stability}}
\label{sec:op_pxi:project_vptstability}

\section{Starting Data Acquisition}
\label{sec:op_pxi:daq_start}

\begin{enumerate}
\item Open the \textit{VPT Stability} project as in \S\ref{sec:op_pxi:project_vptstability}.
\item Open the \menu{Host - Main.vi} VI from the project file viewer.
\item Press the \pgfimage[width=1em,interpolate=true]{figures/labview_button_runonce} \menu{Run Once} button.  You will be prompted for information:
  \begin{itemize}
  \item VPT 1\dashen5 reference numbers
  \item Angle in field (degrees)
  \item Min.\ wait time (seconds)
  \item Load on/off time (hours)
  \end{itemize}
\end{enumerate}

\section{Stopping Data Acquisition}
\label{sec:op_pxi:daq_stop}
\begin{enumerate}
\item If necessary, log into the PXI crate as in \secnameref{sec:op_pxi:rdp}.
\item Locate the \menu{Host - Main.vi} window, listed under the \menu{Window} menu of any LabVIEW window.  The front panel is preferable, but not necessary.
\item Hit the \pgfimage[width=1em,interpolate=true]{figures/labview_button_stop} \menu{Stop} button.
\end{enumerate}

\section{Restarting Data Acquisition}
\label{sec:op_pxi:daq_restart}

Follow this procedure if you were taking data and wish to start over with the same VPTs:
\begin{enumerate}
\item If desired, copy the old data files to a safe location.
\item Delete the original data files.
\item Begin following \secnameref{sec:op_pxi:daq_start}.
\end{enumerate}

\section{Resuming Data Acquisition}
\label{sec:op_pxi:daq_resume}

Follow this procedure if you wish to resume recording data to the same files after an interruption:
\begin{enumerate}
\item If necessary, log into the PXI crate as in \S\ref{sec:op_pxi:rdp}.
\item Locate one of the data files and open it in a text editor.  Copy the first column of the last line.  This is the time offset to resume at.
\item If necessary, start LabVIEW (\S\ref{sec:op_pxi:labview_launch}), open project \textit{VPT Stability} (\S\ref{sec:op_pxi:project_vptstability}), and/or open \menu{Host - Main.vi}.
\item On the top row of the \menu{Host - Main.vi} front panel is a text input box labeled \menu{Test Start Time Offset}.  Click to edit the contents and paste the time offset from step 2.
\item Press the \pgfimage[width=1em,interpolate=true]{figures/labview_button_runonce} \menu{Run Once} button.  When prompted, enter the original VPT numbers, and the rest of the information as before.
\end{enumerate}


\section{Shutting Down The Crate (software)}
\label{sec:op_pxi:shutdown_software}

Follow this procedure if you wish to shut down the PXI Crate to later reboot it:
\begin{enumerate}
\item If necessary, log into the PXI crate as in \secnameref{sec:op_pxi:rdp}.
\item If necessary, shut down DAQ as in \secnameref{sec:op_pxi:daq_stop}.
\item Close LabVIEW.  \FIXME{} menu commands; do not save VIs?
\item \FIXME{} Click the start button and navigate to \menu{Start \menusep Logout}, then choose \menu{Power Off} when prompted.
\end{enumerate}

\section{Powering On Hardware}
\label{sec:op_pxi:poweron}
\begin{enumerate}
\item Locate the power button on the lower left-hand side of the front of the PXI Crate.  Next to the button is an LED light.
\item If the light near the button is lit, the crate is already powered on.  If it is not lit, press the power button.
\end{enumerate}


\section{Powering Down Hardware}
\label{sec:op_pxi:poweroff}
\begin{enumerate}
\item First perform a software shutdown as in \secnameref{sec:op_pxi:shutdown_software}.
\item Check if the power LED is still lit.  It is located on the lower left-hand side of the front of the PXI Crate, near the power button.
\item If still powered, press the power button once.
\end{enumerate}


%%% Local Variables: 
%%% mode: latex
%%% TeX-master: "Manual"
%%% End: 
