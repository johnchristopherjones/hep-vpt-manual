\chapter{Low Voltage Supply}
\label{sec:op_low_voltage}

For operating instructions, including troubleshooting, reference the \href{Manuals/9130_manual.pdf}{BK Precision 9130 User Manual}, or the \href{http://www.bkprecision.com/products/model/9130/triple-output-programmable-dc-power-supply.html}{BK Precision Model 9130 product page}.

\section{Panel Controls}
\label{sec:op_low_voltage:panel_controls}

The \keys{On/Off} key controls the output state (on/off) of all three channels simultaneously.  To control the output state of an individual channel, use the number keys 1\dashen3.

Use the 1\dashen3 keys to set the output state of channels 1\dashen3.  Similarly, use 4\dashen6 keys to set the voltage, and 7\dashen9 keys to set the current for each channel.

\section{Setting Voltage}
\label{sec:op_low_voltage:set_voltage}
There are three different methods to set the voltage:
\begin{enumerate}
\item Press \keys{V-set}.  Enter a numeric value with the keypad, then press \keys{Enter}.
\item Press \keys{V-set}.  Then use the \upkey\downkey\ arrow keys to select a channel.  Adjust the voltage with the knob.
\item Press the \keys{4}, \keys{5}, or \keys{6} key to select channel 1, 2, or 3.  Then enter a numerical value on the keypad.  Then press Enter.
\end{enumerate}

\section{Setting Current}
\label{sec:op_low_voltage:set_current}
There are three different methods to set the current.  They are identical to the methods to set the voltage, except that you press \keys{I-set} instead of \keys{V-set}, and the keys \keys{7}, \keys{8}, or \keys{9} instead of \keys{4}, \keys{5}, or \keys{6}.
\begin{enumerate}
\item Press \keys{I-set}.  Enter a numeric value with the keypad, then press \keys{Enter}.
\item Press \keys{I-set}.  Then use the \upkey\downkey\ arrow keys to select a channel.  Adjust the voltage with the knob.
\item Press the \keys{7}, \keys{8}, or \keys{9} key to select channel 1, 2, or 3.  Then enter a numerical value on the keypad.  Then press Enter.
\end{enumerate}

\section{System Set}
\label{sec:op_low_voltage:system_set}
System Set is a menu available from the \keys{Menu} button.  One of the things it allows you to do is set channels for series or parallel operation.  Supply two should have \menu{Out Serial Set} set to 1+3.  For serial use, Ch1$-$ should be connected to Ch3$+$, and Ch1$+$ and Ch3$-$ should connect to the load.  (Ch 2$+$3 serial operation is not permitted.)

%%% Local Variables: 
%%% mode: latex
%%% TeX-master: "Manual"
%%% End: 
