\section{\glsentryfirst{RNAS}}
\label{sec:eq_readynas}

The \gls{RNAS} is a ready-made NAS solution.  NAS is an acronym for \textit{Network-Attached Storage}, a file-level (as opposed to block-level) remote storage system.  The \href{http://www.netgear.com/Products/Storage/ReadyNASNVPlus.aspx}{NetGEAR ReadyNAS NV+} acts as a network filesystem for the \gls{PXI Crate} in addition to the crate's native filesystem on its local SATA hard drive.

Much of your interaction with the crate will happen indirectly, via the \gls{RNAS}.  You'll usually want to edit VIs locally and then upload them to the \gls{RNAS} when its time to update the experiment's software.  VIs are usually programmed to log their data to the \gls{RNAS}, so you'll retreive the latest data from the \gls{RNAS} as well.

The main exception to this is any VI which requires access to the crate's peripheral hardware, such as the FPGA, DMM, oscilloscope, or switches.  These components need to be programmed and tested from LabVIEW on the PXI Crate itself, as in \secnameref{sec:op_pxi:rdp}.

%The \gls{RNAS} is configured for FTP access, with the username ``\texttt{labview}'' and the password ``\texttt{teststand}''.  
The \gls{RNAS} is configured for FTP access.  
For FTP directions, see \secnameref{sec:op_start:vptvi} and \secnameref{sec:op_start:data}.

\begin{pleasedonot} upload VIs without first making sure that LabVIEW on the crate has closed those VIs.\end{pleasedonot}

\begin{pleasedonot} directly edit VIs or use viewing or processing VIs to view or edit data directly from the \gls{RNAS} if you have chosen to mount the remote filesystem.  You may corrupt LabVIEW state (on the crate or your own computer), or cause availability or timing errors in ongoing experiments.\end{pleasedonot}

\begin{pleasedo} make a local copy of any VI or data you wish to use.  You may safely copy data files while they are being written to.\end{pleasedo}



%%% Local Variables: 
%%% mode: latex
%%% TeX-master: "Manual"
%%% End: 
