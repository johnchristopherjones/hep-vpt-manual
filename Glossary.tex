% \newglossaryentry{<LABEL>}{
%   name=<LATEX>,
%   description=<DEFINITION>,
%   plural=<PLURAL if not just +s>,
%   sort=<term to sort by if not ASCII>}

\newglossaryentry{teststand}{
  name=teststand,
  description={The magnet, mounting brackets, LED pulser boards, and anything else physically attached to the VPTs or the rig}}

\newglossaryentry{PXI Crate}{
  name=PXI Crate,
  description={The National Instruments crate and contents, including hardware modules and software to control the experiment and perform data acquisition (DAQ)}}

\newglossaryentry{break-out board}{
  name=break-out boards,
  description={Break-out boards from the PXI crate which allow connections between the PXI crate and the rig}}


\newglossaryentry{FPGA}{
  name=FPGA,
  first={Field Programmable Gate Array (FPGA)},
  description={}}

\newglossaryentry{voltage supplies}{
  name=voltage supplies,
  description={Low and high voltage supplies}}

\newglossaryentry{rig}{
  name=rig,
  description={The aluminum bit mounted in/on the magnet}}

\newglossaryentry{VPT VI}{
  name=VPT VI,
  description={[``vee-pee-tee vee-eyes''] Literally Vaccum Photo-triode Virtual Instruments; Refers to the HEP software written in LabVIEW for the National Instruments hardware.  Includes software and hardware logic}}

\newglossaryentry{DAQ}{
  name=DAQ,
  description={An abbreviation for Data Acquisition, DAQ refers to the process of capturing digital representations of physical processes.  By definition DAQ involves (typically analog) sensors, circuitry to translate the analog signal into a digitizable form, and an ADC (Analog to Digital Converter).  Colloquially DAQ can also refer to the process of capturing those digital signals and recording them}}

\newglossaryentry{superconducting solenoid}{
  name=superconducting solenoid,
  description={}}

\newglossaryentry{SHVC}{
  name=SHVC,
  description={An acronym for Safe High Voltage (SHV) Connector, it is a connector similar in appearance to a BNC connector, but designed to be safe carrying much higher voltages and currents than traditional BNC connectors.  Like BNC, the connectors are used on co-axial cables.  SHV connectors cannot be mated with low voltage BNC terminals, or vice versa.  SHV connectors are visually distinguishable from BNC connectors by their thick protruding insulator.  They are designed so that high voltage contact is broken before the grounding contact is, protecting the user from shock}}

\newglossaryentry{BNC}{
  name=BNC,
  description={}}

\newglossaryentry{MOLEX}{
  name=MOLEX,
  description={}}

\newglossaryentry{attenuators}{
  name=attenuators,
  description={}}

\newglossaryentry{LabVIEW}{
  name=LabVIEW,
  description={}}

\newglossaryentry{pre-amp board}{
  name=pre-amp board,
  description={}}

\newglossaryentry{LED board}{
  name=LED board,
  description={}}

\newglossaryentry{LED control}{
  name=LED control,
  description={}}

\newglossaryentry{RNAS}{
  name=RNAS,
  first={ReadyNAS (RNAS)},
  description={ReadyNAS, a specific NAS product produced by Netgear.  RNAS is a specific independent hardware module located in the HEP Computer Room}}

\newglossaryentry{NAS}{
  name=NAS,
  description={An acronym for Network-Attached Storage, it refers to a computer file-system accessible by heterogenous clients.  ReadyNAS (RNAS) provides a NAS solution, which can be accessed by SMB (Windows File Sharing), FTP, and RSYNC clients}}

\newglossaryentry{HEP}{
  name=HEP,
  description={An acronym for High-Energy Physics.  At UVA, HEP is a field of research with specific faculty.  Research in high energy physics is based in the High Energy Physics Building, located [blank].  Many of the faculty and graduate students working in high-energy physics, especially experimentalists, have offices in the HEP building}}
