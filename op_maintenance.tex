\chapter{Maintainence}

\section{Schedule}
\label{sec:op_maintenance}

This section lists tasks which must be done regularly to maintain the experimental equipment, or ongoing experiments.  The following vocabulary is used in this section.
\begin{table}[h]\begin{tabular}{>{\scshape}r p{\textwidth}}
    daily & Once per day, at any time unless otherwise specified\\
    semi-daily & Every other day\\
    biweekly & Twice a week, or every 3-4 days\\
    monthly & Once per month\\
    as needed & As often as necessary; frequency determined by another maintenance step\\
\end{tabular}\end{table}

\subsection{Under All Conditions}
\label{sec:op_maintenance:always}
The following tasks must be carried out whether or not an experiment is currently under way.  

\begin{table}[h]\begin{tabular}{>{\scshape}r p{\textwidth}}
    daily & Measure cryogen levels\\
    as needed & Fill LN2 cryogen\\
    as needed & Fill LHe cryogen\\
    monthly & Measure magnetic field strength    
\end{tabular}\end{table}

\subsection{Experiment: VPT Stability}
\label{sec:op_maintenance:ex_vpt_stability}
The following tasks are only required during VPT Stability experiments.

\begin{table}[h]\begin{tabular}{>{\scshape}r p{\textwidth}}
    daily & Verify DAQ is still running\\
    biweekly & Examine data for experimental errors
\end{tabular}\end{table}

\section{Measuring Cryogen Levels}
\label{sec:op_maintenance:cryogen_levels}

Cryogen levels should be checked daily.  Under normal conditions the cryogen evaporation rate is virtually constant.  However, checking daily will reveal if a fill was done improperly, or if a \gls{quench} occured.  

\begin{enumerate}
\item Locate the cryogen lab notebook near the cryogen guages.
\item Record the current date and time in the notebook.
\item Read the liquid nitrogen guage, which is always on.  Record the measurement in the notebook.
\item To begin taking a liquid helium measurement, press the green power button to turn on the guage.
\item Wait several seconds, then press the black ``MAN'' button to take a measurement.  The ``Sample'' light will light up.
\item Wait until the ``Sample'' light goes out, then read the measurement from the LCD display.  It's a percentage.
\item Record the LHe measurement in the notebook.
\item Press the green power button to turn off the LHe guage.
\end{enumerate}

\begin{pleasedonot}
  leave the liquid helium guage powered on.  It will unnecessarily heat the cryogens and cause them to boil off more rapidly.
\end{pleasedonot}

\section{Filling LN2 Cryogen}
\label{sec:op_maintenance:filling_ln2}
\begin{pleasedo} consider filling Monday and Friday, and always well before reaching 10~\% capacity. \end{pleasedo}
\FIXME Placeholder for practiced fill
\begin{pleasedo} move the empty dewar through the computer room and out the doors to the concrete patio. \end{pleasedo}

\section{Filling LHe Cryogen}
\label{sec:op_maintenance:filling_lhe}
\begin{pleasedo} fill between 20\dashen30~\% capacity to use an entire LHe dewar. \end{pleasedo}
\FIXME Placeholder for practiced fill
\begin{pleasedo} move the empty dewar through the computer room and out the doors to the concrete patio. \end{pleasedo}


%%% Local Variables: 
%%% mode: latex
%%% TeX-master: "Manual"
%%% End: 
