\chapter{Maintainence}
\label{sec:op_maintenance}

\section{Schedule}
\label{sec:op_maintenance:schedule}

This section lists tasks which must be done regularly to maintain the experimental equipment or ongoing experiments.  The following vocabulary is used in this section:
\begin{table}[h]\begin{tabular}{>{\scshape}r p{\textwidth}}
    daily & Once per day, at any time unless otherwise specified\\
    semi-daily & Every other day\\
    biweekly & Twice a week, or every 3-4 days\\
    monthly & Once per month\\
    as needed & As often as necessary; frequency determined by another maintenance step\\
\end{tabular}\end{table}

\subsection{Under All Conditions}
\label{sec:op_maintenance:always}
The following tasks must be carried out whether or not an experiment is currently under way.  

\begin{table}[h]\begin{tabular}{>{\scshape}r p{\textwidth}}
    daily & Measure cryogen levels\\
    as needed & Fill LN2 cryogen\\
    as needed & Fill LHe cryogen\\
    monthly & Measure magnetic field strength    
\end{tabular}\end{table}

\subsection{Experiment: VPT Stability}
\label{sec:op_maintenance:ex_vpt_stability}
The following tasks are only required during VPT Stability experiments.

\begin{table}[h]\begin{tabular}{>{\scshape}r p{\textwidth}}
    daily & Verify DAQ is still running\\
    biweekly & Examine data for experimental errors
\end{tabular}\end{table}

\section{Measuring Cryogen Levels}
\label{sec:op_maintenance:cryogen_levels}

Cryogen levels should be checked daily.  Under normal conditions the cryogen evaporation rate is virtually constant.  However, checking daily will reveal if a fill was done improperly, or if a \gls{quench} occured.  

\begin{enumerate}
\item Locate the cryogen lab notebook near the cryogen gauges.
\item Record the current date and time in the notebook.
\item Read the liquid nitrogen gauge, which is always on.  Record the measurement in the notebook.
\item To begin taking a liquid helium measurement, press the green power button to turn on the gauge.
\item Wait several seconds, then press the black ``MAN'' button to take a measurement.  The ``Sample'' light will light up.
\item Wait until the ``Sample'' light goes out, then read the measurement from the LCD display.  It's a percentage.
\item Record the LHe measurement in the notebook.
\item Press the green power button to turn off the LHe gauge.
\end{enumerate}

\begin{pleasedonot}
  leave the liquid helium gauge powered on.  It will unnecessarily heat the cryogens and cause them to boil off more rapidly.
\end{pleasedonot}

\section{Filling LN2 Cryogen}
\label{sec:op_maintenance:filling_ln2}
\begin{pleasedo} consider filling Monday and Friday, and always well before reaching 10~\% capacity. \end{pleasedo}

\begin{enumerate}
\item Measure and log the cryogen levels, as in \S\ref{sec:op_maintenance:cryogen_levels}.
\item Climb up the ladder and unscrew the wingnut from the c-clamp at the base of the black ventilation tower.
\item Remove the c-clamp, ventilation tower, \textbf{and the o-ring} beneath the tower.
\item Climb down and slowly turn the blue valve (connected by pipe to the magnet).  Allow the LN2 to flow slowly at first to cool the valve and piping, then open the valve all the way.  A constant plume of white vapour will shoot from the valve where the ventilation tower was removed.
\item Return to the LN2 gauge and monitor the fill. It takes 10\,min on average to fill 25\,\%.
\item Dust frost off the ventillation tower valve every 5\dashen10\,min or so.
\item Once the gauge reaches 100\,\%, return to the LN2 dewar and shut off the blue valve.
\item Climb up the ladder and thoroughly clean the tower valve.
\item Replace \textbf{the o-ring}, ventillation tower, and re-attach the c-clamp.
\item Firmly tighten the wingnut on the c-clamp by hand.
\item Return to the cryogen gauges and record the 100\,\% LN2 level, as in \S\ref{sec:op_maintenance:cryogen_levels}.
\end{enumerate}

\begin{pleasedonot} forget to replace the o-ring.  Failing to replace the o-ring is the easiest mistake to make during an LN2 fill and will cause LN2 to boil off more rapidly. \end{pleasedonot}

\begin{pleasedo} move the empty dewar through the computer room and out the doors to the concrete patio. \end{pleasedo}

\section{Ordering LN2 Cryogen}
\label{sec:op_maintenance:ordering_ln2}

\FIXME{} Chris in the stock room in the Beams building handles orders.

\section{Filling LHe Cryogen}
\label{sec:op_maintenance:filling_lhe}
\begin{pleasedo} fill between 20\dashen30~\% capacity to use an entire LHe dewar. \end{pleasedo}
\FIXME Placeholder for practiced fill\\
\begin{pleasedo} move the empty dewar through the computer room and out the doors to the concrete patio. \end{pleasedo}

\section{Ordering LHe Cryogen}
\label{sec:op_maintenance:ordering_lhe}

Mike (HEP) handles orders.  Takes 2\dashen3 weeks.

%%% Local Variables: 
%%% mode: latex
%%% TeX-master: "Manual"
%%% End: 
