\pgfmathparse{0.2ex*1em/1ex}
\tikzset{pxislot/.style={line width=\pgfmathresult, baseline=0.34em/2}}

\newcommand{\pxislotone}{\tikz[baseline=(n.base)]\draw[rotate=180] (0,0)--(60:1.4em)--(120:1.4em)--cycle node (n) at (0,.707em) {\tiny 1};}
\newcommand{\pxislottwo}{\tikz[baseline=(n.base),scale=.75]\draw[rotate=45] (0,0) circle (.6em) (-.6em,-.6em) rectangle (.6em,.6em) node (n) at (0,0) {\tiny 2};}
\newcommand{\pxislotn}[1]{\tikz[baseline=(n.base)]\draw (0,0) circle (.6em) node (n) {\tiny #1};}

% \newcommand{\pxislotone}{\noindent
%   \beginpgfgraphicnamed{Graphics-pxislot1}
%   \begin{tikzpicture}[pxislot]
%     \draw (0,0) -- (0.5em,0.866em) -- (1em,0) -- (0,0) -- cycle;
%     \pgfusepath{use as bounding box}
%     \node[scale=0.8] at (0.5em,0.34em) {\texttt{1}};
%   \end{tikzpicture}
%   \endpgfgraphicnamed
% }

% \newcommand{\pxislottwo}{\noindent
%   \beginpgfgraphicnamed{Graphics-pxislot2}
%   \begin{tikzpicture}[pxislot]
%     \draw (0.5em,0.5em) circle (0.355em);
%     \draw (0,0.5em) -- (0.5em,1em) -- (1em,0.5em) -- (0.5em,0) -- cycle;
%     \node[scale=0.8] at (0.5em,0.5em) {\texttt{2}};
%   \end{tikzpicture}
%   \endpgfgraphicnamed
% }

% \newcommand{\pxislotn}[1]{\noindent
%   \beginpgfgraphicnamed{Graphics-pxislot#1}
%   \begin{tikzpicture}[pxislot]
%     \draw (0.5em,0.5em) circle (0.5em);
%     \node[scale=0.8] at (0.5em,0.5em) {\texttt{#1}};
%   \end{tikzpicture}
%   \endpgfgraphicnamed
% }

% \pxislotone
% \pxislottwo
% \foreach \x in {3,...,7}
% {
%   \pxislotn{\x}
% }


%%% Local Variables: 
%%% mode: latex
%%% TeX-master: "Manual"
%%% End: 
