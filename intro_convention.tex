\section{Conventions Used \textcolor{red}{DRAFT}}

The following conventions are used in this text.

\begin{description}

  \item[\menu{File \menusep Open}] For menu items, a sans-serif font is
    used with \menusep between the menu items.
  \item[\keys{keys}] For short key sequences that sould be pressed, a sans-serif font is used.
  \item[\path{/foo/bar}] For directories, filenames, and paths, a mono-spaced font is used.
  \item[\command{command -options filename.ext}] For command which should be
    entered literally into a terminal, a bold mono-spaced font is used.
  \item[\command{--filename \namedfield{named field}}] For options the user should supply,
    a brief description of the option is surrounded in angle brackets.
  \item[\noun{LabVIEW}] For proper nouns of software, like application names and operating systems, a sans-serif font is used. (\FIXME An italic or normal font may improve readability.)
  \item[\makemodel{CAEN}{SY1527LC}] The make (manufacturer) and model number of a component are typeset this way.
  \item[\make{CAEN}] Just the maker of a component is typeset this way.
  \item[\model{SY1527}] Just the model number of a component is typeset this way.

\end{description}


\begin{avoid} hazards pointed out by the warning signs, unless duly considered. \end{avoid}
\begin{pleasedo} read positive recommendations in boxes like this. \end{pleasedo}
\begin{pleasedonot} ignore negative recommendations without due consideration. \end{pleasedonot}

\section{Symbols Used}
\label{sec:intro:symbols}

\subsection{Keyboard Symbols}
\label{sec:intro:keyboard_symbols}
For brevity and consistency, a number of standard symbols are used to represent keyboard keys.  These conventions were largely adopted from \noun{Mac~OS~X}.

% Glyph addresses taken from:
% http://www.macosxhints.com/article.php?story=20071028203517911
% OS X 10.6 uses Lucida Grande for these symbols
\newcommand{\shiftkey}      {\textsf{⇧}} % U+21E7
\newcommand{\controlkey}    {\textsf{⌃}} % U+2303
\newcommand{\optionkey}     {\textsf{⌥}} % U+2325
\newcommand{\commandkey}    {\textsf{⌘}} % U+2318
\newcommand{\deleterightkey}{\textsf{⌦}} % U+2326
\newcommand{\deleteleftkey} {\textsf{⌫}} % U+232B
\newcommand{\escapekey}     {\textsf{⎋}} % U+238B
\newcommand{\returnkey}     {\textsf{↩}} % U+21A9
\newcommand{\leftkey}       {\textsf{←}} % U+2190
\newcommand{\upkey}         {\textsf{↑}} % U+2191
\newcommand{\rightkey}      {\textsf{→}} % U+2192
\newcommand{\downkey}       {\textsf{↓}} % U+2193
\newcommand{\tabkey}        {\textsf{⇥}} % U+21E5
\begin{table}[htbp]\begin{center}
    \caption{Keyboard Symbols}
    \label{tab:intro:keyboard_symbols}

    \begin{tabular}{@{} c l l @{}}
      \toprule %%%%%%%%%%%%%%%%%%%%%%%%%%%%%%%%%%%%%%%%%%%%%%%%%%%%%%%%%%%%%%%%%
      Symbol          & Name            & Also known as \\
      \midrule %%%%%%%%%%%%%%%%%%%%%%%%%%%%%%%%%%%%%%%%%%%%%%%%%%%%%%%%%%%%%%%%%
      \shiftkey       & Shift           & \textit{Shift key}\\
      \controlkey     & Control         & \textit{Control key}\\
      \optionkey      & Option          & On Windows: \textit{Alt key}\\
      \commandkey     & Command         & On Windows: \textit{Windows key}\\
      \deleterightkey & Delete Right    & \textit{Delete key}\\
      \deleteleftkey  & Delete Left     & \textit{Backspace key}.\\
      \escapekey      & Escape          & \textit{Escape key}\\
      \returnkey      & Return          & Sometimes: \textit{Enter key}\\
      \leftkey        & Left            & \textit{Left-arrow key}\\
      \upkey          & Up              & \textit{Up-arrow key}\\
      \rightkey       & Right           & \textit{Right-arrow key}\\
      \downkey        & Down            & \textit{Down-arrow key}\\
      \tabkey         & Tab             & \textit{Tab key}\\
      \bottomrule %%%%%%%%%%%%%%%%%%%%%%%%%%%%%%%%%%%%%%%%%%%%%%%%%%%%%%%%%%%%%%
\end{tabular}
\end{center}\end{table}


Four of these keys are \textit{modifiers}: \commandkey , \shiftkey , \controlkey , \optionkey .  These keys do nothing on their own (except for \commandkey\  under Windows, which toggles the \textsf{Start Menu}), and have to be combined with another character.  This is denoted by joining two keys, such as \keys{\commandkey C} (Copy, \textsf{OS~X}) or  \keys{\controlkey C} (Copy, \textsf{Windows}).

Should it be necessary to describe a \textit{key chord}, which is a sequence of keys which are pressed and released, the individual strokes will be separated by spaces.  For instance, the \textsf{Save~As} command in \textsf{Emacs} can be called by holding \textit{control} while pressing and releasing \texttt{\textbf{x}}, pressing and releasing \texttt{\textbf{s}}, then releasing \textit{control}.  Or, in other words: \keys{\controlkey x \controlkey s}

%%% Local Variables: 
%%% TeX-master: "Manual"
%%% End: 

